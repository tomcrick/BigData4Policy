\documentclass[a4paper,11pt]{article}
\usepackage[top=1.8cm,bottom=1.8cm,left=1.8cm,right=1.8cm,asymmetric]{geometry}
\usepackage{url}
\usepackage{paralist}
\usepackage{authblk}
\usepackage[sort&compress]{natbib}
\usepackage[pdftex,colorlinks=true,hyperfootnotes=false]{hyperref}

\title{\vspace{-3em}Innovation Policy-Making in the Big Data Era}

\author{ }
% \author[1]{Tom Crick\thanks{Tom Crick acknowledges support as Nesta's Data Science Fellow}}
% \author[2]{Juan Mateos-Garcia}
% \author[2]{Hasan Bakhshi}
% \author[2]{Stian Westlake}
% \affil[1]{Cardiff Metropolitan University}
% \affil[2]{Nesta}
% \affil[1]{\protect\url{tcrick@cardiffmet.ac.uk}}
% \affil[2]{\protect\url{{juan.mateos-garcia,hasan.bakhshi,stian.westlake}@nesta.org.uk}}

% \renewcommand\Authands{ and }
% \def\UrlBreaks{\do\/\do-}

\date{ }

\begin{document}

\maketitle

\vspace{-4em}
% \begin{abstract}
% \end{abstract}

% 1000 word extended abstract

There has been an explosion of interest in the potential of big data
as a driver of better decision-making in many policy areas, including
innovation policy.  In this paper, we draw on the theory of
innovation, and on Nesta's own
experience\footnote{e.g. \url{http://www.nesta.org.uk/blog/big-data-better-innovation-policy}}
as an innovation agency in order to address the following research
questions:

\begin{itemize}
\item {\emph{What are the characteristics of innovation policy that make it a
  suitable domain for the application of big data sources and
  analytical methodologies?}}
\item {\emph{What is the state of the art, and what are the emerging
  opportunities for the application of big data for innovation policy?}}
\end{itemize}

In doing this, we seek to provide a firmer conceptual grounding for
work in this area, and to set a vision for the development of big data
applications addressing the needs of innovation policymakers.

We begin by identifying the main rationales for innovation policy:
{\emph{market failures}} linked to the fact that innovators often fail to
fully capture the benefits of their investments, {\emph{systems failure}}
caused by gaps in the ``system of innovation'' that ought to connect
innovation agents, and {\emph{inhibited emergence}}, where a state of
uncertainty about the future configuration of a market or technology
field hinders its development~\citep{gustafsson+autio:2011}.

There are many policy options to remove these barriers to innovation,
ranging from direct public investments on R\&D to regulation and
procurement~\citep{edler-et-al:2013}. Their design, implementation and
evaluation has traditionally been based on data sources such as
business and innovation surveys, administrative data, and metrics of
scientific and technological output (academic publications and
patents)~\citep{fagerberg-et-al:2006}. Three defining characteristics
of innovation do however limit the usefulness of these data sources
and outputs for innovation policymaking:

\begin{enumerate}
\item {\textbf{Innovation involves novelty in inputs, processes and outputs:}} it
  is associated with new capabilities, forms of organisation and
  industries which, by definition, are not captured by existing
  classifications of economic activity such as Standard Occupational
  Classifications (SOC) and Standard Industrial Classifications
  (SIC).
\item {\textbf{Innovation is not confined to science and technology:}} it may
  reflect changes in, say, business model, marketing or aesthetic: as
  a result, it is not always captured by traditional metrics such as
  academic papers and patents. 
\item {\textbf{Innovation is a complex, networked process:}} it reflects a
  dynamic combination of resources and capabilities of many different
  agents and institutions. Measuring it requires combining data from a
  multitude of these sources. In turn, those who stand to benefit from
  access to data on innovation goes beyond policymakers, and
  encompasses investors, entrepreneurs and corporates, to name a
  few. However, in practice, most (aggregated, lagging) innovation
  data outputs are of limited relevance for these agents.
\end{enumerate}

Big data can help overcome some of these challenges for innovation
policymaking using conventional data inputs and outputs. Following
~\citet{schroeder+cowls:2014}, we define big data as datasets of a
volume, variety (complexity) and velocity unprecedented in the
innovation policy domain, together with new analytical techniques and
data outputs (such as data visualisations and interactive data
platforms) used to analyse and create value from these data.

Big datasets (e.g. information provided by businesses on their
websites) are often unstructured and closer to real-time than official
data. This means that they can be used to identify new innovation
areas as they emerge, even when these do not respect traditional
occupational or industry boundaries. Some metrics that policymakers
use to measure innovation are steeped in scientific and technological
understandings of innovation; big datasets, by expanding possible
sources of data, need not be so constrained. Big data is
high-resolution (when it is based on publicly available data, it is
often possible to identify individual agents like businesses or
investors in it in ways that official data, which is subject to
non-disclosure constraints, is not). This makes it easier to republish
it in interactive formats, say, that can be queried and exploited by a
variety of innovation agents in addition to policymakers (this is
manifest in the recent creation of a variety of online platforms to
map innovative industries, clusters and ecosystems using, for example,
publicly available data from Companies House).

Big data is however no panacea for policymakers, and its use in
innovation policy presents serious methodological
challenges. Nevertheless, there exists significant policy work done in
the open science space in which to analyse and
leverage~\citep{rssaaoe:2012}. The innovation landscape is constantly
shifting, leading to the arrival of new data sources to study, and
structural changes that can impact the reliability of algorithmic data
collection and analysis. There is a risk that online data sources
might offer a biased representation of innovation activity that
privileges digitised industries at the expense of those trading in
physical goods and services, and consumer facing industries at the
expense of business-to-business sectors. Privacy is of course another
critical consideration, especially where personal information is
involved~\citep{hocstc:2014}, but there is an imperative for open and
sharable data to provide the platform for effective (and transparent)
policy-making.

Notwithstanding these important issues, we consider that there is
significant scope for innovation in the use of big data for innovation
policy -- we conclude this paper by outlining some of the main
opportunities, and setting out Nesta's future programme of research
and platform development in this space:

\begin{enumerate}
\item Go beyond innovation maps based on SOC and SIC coding and
  official geographies, and make more use of unstructured data
  collection and (supervised and unsupervised) classification methods.
\item Complement descriptive analyses and sample-based
  inference with predictive modelling.
\item Explore the opportunities of real-time data and interactive
  data visualisation for innovation policy.
\item Combine big data sources with official and policy activity
  data in order to evaluate innovation policy impacts.
\item Develop standards for data-sharing so as to minimise the risk
  of fragmentation into incompatible platforms capturing disparate
  aspects of innovation activity.
\item Develop open datasets and transparent (as compared to ``black
  box'') methodologies for big data analysis.
\item Creatively explore the potential use of big data sources and
methods for the study of industries which may not currently be data
science-literate -- and therefore less well catered for by online data
-- but are of great importance for policymakers, such as
manufacturing.
\end{enumerate}


\bibliographystyle{unsrtnat}
\bibliography{BD4P}

\end{document}
